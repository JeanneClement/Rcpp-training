%\usepackage{pslatex} % to see correctly a .pdf file on the computer screen
\usepackage{pgf}
\usepackage{color}
\usepackage{graphicx}
\usepackage{amssymb} %symbole de maths
\usepackage{amsmath} %idem
\usepackage[utf8]{inputenc}
%\usepackage{fancyvrb} %give size to verbatim
%\usepackage{hyperref}
\usepackage[english,francais]{babel}
\definecolor{vertmoyen}{RGB}{51,100,23} % vert moyen
\usecolortheme[named=vertmoyen]{structure}

\usepackage{tabularx} % varier la largeur du tableau
\usepackage{layout}
\usepackage{longtable}
\setlength{\LTleft}{-5cm plus 1 fill}
\setlength{\LTright}{-5cm plus 1 fill}
\usepackage{booktabs}
\usepackage{arydshln} %% dashlines for tabular
\newcommand{\logit}{\text{logit}}
\newcommand{\bs}[1]{\boldsymbol{#1}}
\newcommand{\R}{\textnormal{\sffamily\bfseries R}}
\newcommand{\pkg}[1]{{\fontseries{b}\selectfont #1}}
\newcolumntype{C}[1]{>{\centering\arraybackslash}m{#1}}
%% Natbib is a popular style for formatting references.
%\usepackage{natbib} %doesn't work with beamer

\title[Rcpp* for MCMC in R]{\textbf{Using Rcpp* packages for easy and fast extension of R with C++}}
%\subtitle{} 
%\subtitle{} 
%\author{G. Vieilledent}
%\institute{JRC, CIRAD}
\date{}

% Theme
% \usetheme{AnnArbor}
% \usetheme{Dresden}
% \usetheme{Copenhagen}
% \usetheme{Frankfurt}
% \usetheme{Berlin}
% \usetheme{Madrid}
 \usetheme{Montpellier}
% \usetheme{Singapore}
% \usetheme{Antibes}
\useinnertheme{rounded} %% bullets
\useoutertheme[subsection=false]{miniframes}
%\useoutertheme[subsection=false,footline=authorinstitutetitle]{miniframes}
\setbeamertemplate{footline}[frame number]

% Ignore ignorenonframetext class option in default template
\makeatletter
\beamer@ignorenonframefalse
\makeatother

% Logo
\newif\ifplacelogo % create a new conditional
\logo{
  \ifplacelogo
    \begin{tabular}{C{2cm}C{2cm}C{5cm}}
      \includegraphics[height=0.7cm]{figs/logo_Cirad.png} &
      \includegraphics[height=1cm]{figs/logo_AMAP.png} 
      \includegraphics[height=0.8cm]{figs/AMAP-titre-long.png} &
      ~
    \end{tabular}
  \fi
} % replace with your own command

%Call table of contents at the beginning of each section
\AtBeginSection[]{}
% \AtBeginSection[]{
% \placelogotrue
%   \begin{frame}
%     \frametitle{Outline}
%     \begin{columns}[c]
%       \begin{column}{0.5\textwidth}
%         \tableofcontents[sections=1, currentsection]
%         \vspace{0.5cm}
%         \tableofcontents[sections=2, currentsection]
%       \end{column}
%       \begin{column}{0.5\textwidth}
%         \tableofcontents[sections=3, currentsection]
%       \end{column}
%     \end{columns}
%   \end{frame}
% \placelogofalse
% }

\AtBeginSubsection[]{}
% \AtBeginSubsection[]{
% \placelogotrue
%   \begin{frame}
%     \frametitle{Plan}
%     \begin{columns}[c]
%       \begin{column}{0.5\textwidth}
%         \tableofcontents[sections={1-2},currentsection,currentsubsection]
%       \end{column}
%       \begin{column}{0.5\textwidth}
%         \tableofcontents[sections={3-4},currentsection,currentsubsection]
%       \end{column}
%     \end{columns}
%   \end{frame}
% \placelogofalse
% }

% Two-columns slides
\def\bcols{\begin{columns}}
\def\bcol{\begin{column}}
\def\ecol{\end{column}}
\def\ecols{\end{columns}}

% Change fontsize of R code
\let\oldShaded\Shaded
\let\endoldShaded\endShaded
\renewenvironment{Shaded}{\footnotesize\oldShaded}{\endoldShaded}

% Change fontsize of output
\let\oldverbatim\verbatim
\let\endoldverbatim\endverbatim
\renewenvironment{verbatim}{\footnotesize\oldverbatim}{\endoldverbatim}
